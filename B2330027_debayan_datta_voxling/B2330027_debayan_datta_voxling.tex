\documentclass{article}
\usepackage{graphicx}
\usepackage{hyperref}
\usepackage{geometry}
\geometry{a4paper, margin=0.8in}
\title{Hindi Speech to English Text Translation System}
\author{VOXLING\\
Debanjan Nanda\hspace{0.2cm} Debayan Datta\hspace{0.2cm} Ayan Maity}
\date{\today}

\begin{document}

\maketitle

\section*{Abstract}
The VOXLING project develops a Hindi Speech to English Text Translation System that converts Hindi audio to Hindi text and then translates it into English. This system aims to enhance accessibility for non-native Hindi speakers and facilitate communication for Hindi speakers in non-Hindi-speaking regions. By leveraging advanced speech recognition and machine translation technologies, VOXLING addresses real-time translation needs and advances research in low-resource languages.

\section*{Introduction}
VOXLING is designed to bridge language barriers by converting Hindi audio into Hindi text and subsequently translating it into English. This system supports better communication for non-English speakers and aids in accessing content in Hindi. It also contributes to research and development in multilingual translation technologies.

\section*{Methodology}
\subsection*{What ?}
\textbf{Model 1: Hindi Speech to Hindi Text}
\begin{itemize}
    \item Dataset Source:\href{https://www.openslr.org/103/}{OpenSLR}
\end{itemize}
\vspace{0.1cm}
\textbf{Model 2: Hindi Text to English Text}
\begin{itemize}
    \item Dataset Source: \href{https://www.cfilt.iitb.ac.in/iitb_parallel/}{IITB English-Hindi Corpus}
\end{itemize}
\subsection*{Why?}

\begin{itemize}
    \item \textbf{Language Barrier Reduction}: Bridges the gap for non-English speakers and enhances real-time translation.
    \item \textbf{Accessibility}: Facilitates access to Hindi content for non-native speakers.
    \item \textbf{Common Ground for Communication}: Assists native Hindi speakers in non-Hindi-speaking regions.
    \item \textbf{Automated Translation System}: Useful in industries like Tourism, Service Centers, and Media.
    \item \textbf{Research \& Development}: Promotes R\&D in Hindi, a low-resource language.
\end{itemize}

\subsection*{How ?}
\begin{itemize}
    \item \textbf{Model 1:} Takes the Hindi Audio Input from Dataset 1, which contains ID, Hindi Speech, Hindi Text and English Text (collected manually by using Google Translate and considered as ground truth). The model processes the Hindi Speech to predict an intermediate Hindi Text output. ~\cite{jia2022cvss, wang2020covost}
    \item \textbf{Model 2:} Uses Dataset 2, consisting of Hindi Text and English Text. It takes the predicted Hindi Text from Model 1 and translates it into English Text, producing the final Predicted English Text Output.~\cite{gangar2021hindi}
    \item \textbf{Comparison:} The final output is compared with the actual English Text from Dataset 1 to evaluate the model's performance.
\end{itemize}
\newpage
\section*{Work Plan \& Timeline}
\begin{table}[h!]
    \centering
    \begin{tabular}{|l|l|l|}
        \hline
        \textbf{Task} & \textbf{Start Date} & \textbf{End Date} \\
        \hline
        Data Collection & 20-08-2024 & 26-08-2024 \\
        Literature Survey & 27-08-2024 & 06-09-2024 \\
        Data Preprocessing & 01-09-2024 & 11-09-2024 \\
        Model 1 Training & 12-09-2024 & 20-10-2024 \\
        Model 2 Training & 24-09-2024 & 20-10-2024 \\
        Evaluation and Validation & 20-10-2024 & 27-10-2024 \\
        Error Analysis and Optimization & 27-10-2024 & 02-11-2024 \\
        System Integration & 02-11-2024 & 10-11-2024 \\
        Final Testing and Validation & 10-11-2024 & 19-11-2024 \\
        \hline
    \end{tabular}
    \caption{Timeline of the Project}
\end{table}

\section*{Work Division}
We plan on working collaboratively and will divide tasks based on each other's strengths and weaknesses.


\nocite{*}% Biblography without citation!
	
\bibliography{dl_nlp_reference} % .bib file
\bibliographystyle{plain}

\end{document}